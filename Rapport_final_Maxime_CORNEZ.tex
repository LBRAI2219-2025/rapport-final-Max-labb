% Options for packages loaded elsewhere
\PassOptionsToPackage{unicode}{hyperref}
\PassOptionsToPackage{hyphens}{url}
%
\documentclass[
]{article}
\usepackage{amsmath,amssymb}
\usepackage{iftex}
\ifPDFTeX
  \usepackage[T1]{fontenc}
  \usepackage[utf8]{inputenc}
  \usepackage{textcomp} % provide euro and other symbols
\else % if luatex or xetex
  \usepackage{unicode-math} % this also loads fontspec
  \defaultfontfeatures{Scale=MatchLowercase}
  \defaultfontfeatures[\rmfamily]{Ligatures=TeX,Scale=1}
\fi
\usepackage{lmodern}
\ifPDFTeX\else
  % xetex/luatex font selection
\fi
% Use upquote if available, for straight quotes in verbatim environments
\IfFileExists{upquote.sty}{\usepackage{upquote}}{}
\IfFileExists{microtype.sty}{% use microtype if available
  \usepackage[]{microtype}
  \UseMicrotypeSet[protrusion]{basicmath} % disable protrusion for tt fonts
}{}
\makeatletter
\@ifundefined{KOMAClassName}{% if non-KOMA class
  \IfFileExists{parskip.sty}{%
    \usepackage{parskip}
  }{% else
    \setlength{\parindent}{0pt}
    \setlength{\parskip}{6pt plus 2pt minus 1pt}}
}{% if KOMA class
  \KOMAoptions{parskip=half}}
\makeatother
\usepackage{xcolor}
\usepackage[margin=1in]{geometry}
\usepackage{color}
\usepackage{fancyvrb}
\newcommand{\VerbBar}{|}
\newcommand{\VERB}{\Verb[commandchars=\\\{\}]}
\DefineVerbatimEnvironment{Highlighting}{Verbatim}{commandchars=\\\{\}}
% Add ',fontsize=\small' for more characters per line
\usepackage{framed}
\definecolor{shadecolor}{RGB}{248,248,248}
\newenvironment{Shaded}{\begin{snugshade}}{\end{snugshade}}
\newcommand{\AlertTok}[1]{\textcolor[rgb]{0.94,0.16,0.16}{#1}}
\newcommand{\AnnotationTok}[1]{\textcolor[rgb]{0.56,0.35,0.01}{\textbf{\textit{#1}}}}
\newcommand{\AttributeTok}[1]{\textcolor[rgb]{0.13,0.29,0.53}{#1}}
\newcommand{\BaseNTok}[1]{\textcolor[rgb]{0.00,0.00,0.81}{#1}}
\newcommand{\BuiltInTok}[1]{#1}
\newcommand{\CharTok}[1]{\textcolor[rgb]{0.31,0.60,0.02}{#1}}
\newcommand{\CommentTok}[1]{\textcolor[rgb]{0.56,0.35,0.01}{\textit{#1}}}
\newcommand{\CommentVarTok}[1]{\textcolor[rgb]{0.56,0.35,0.01}{\textbf{\textit{#1}}}}
\newcommand{\ConstantTok}[1]{\textcolor[rgb]{0.56,0.35,0.01}{#1}}
\newcommand{\ControlFlowTok}[1]{\textcolor[rgb]{0.13,0.29,0.53}{\textbf{#1}}}
\newcommand{\DataTypeTok}[1]{\textcolor[rgb]{0.13,0.29,0.53}{#1}}
\newcommand{\DecValTok}[1]{\textcolor[rgb]{0.00,0.00,0.81}{#1}}
\newcommand{\DocumentationTok}[1]{\textcolor[rgb]{0.56,0.35,0.01}{\textbf{\textit{#1}}}}
\newcommand{\ErrorTok}[1]{\textcolor[rgb]{0.64,0.00,0.00}{\textbf{#1}}}
\newcommand{\ExtensionTok}[1]{#1}
\newcommand{\FloatTok}[1]{\textcolor[rgb]{0.00,0.00,0.81}{#1}}
\newcommand{\FunctionTok}[1]{\textcolor[rgb]{0.13,0.29,0.53}{\textbf{#1}}}
\newcommand{\ImportTok}[1]{#1}
\newcommand{\InformationTok}[1]{\textcolor[rgb]{0.56,0.35,0.01}{\textbf{\textit{#1}}}}
\newcommand{\KeywordTok}[1]{\textcolor[rgb]{0.13,0.29,0.53}{\textbf{#1}}}
\newcommand{\NormalTok}[1]{#1}
\newcommand{\OperatorTok}[1]{\textcolor[rgb]{0.81,0.36,0.00}{\textbf{#1}}}
\newcommand{\OtherTok}[1]{\textcolor[rgb]{0.56,0.35,0.01}{#1}}
\newcommand{\PreprocessorTok}[1]{\textcolor[rgb]{0.56,0.35,0.01}{\textit{#1}}}
\newcommand{\RegionMarkerTok}[1]{#1}
\newcommand{\SpecialCharTok}[1]{\textcolor[rgb]{0.81,0.36,0.00}{\textbf{#1}}}
\newcommand{\SpecialStringTok}[1]{\textcolor[rgb]{0.31,0.60,0.02}{#1}}
\newcommand{\StringTok}[1]{\textcolor[rgb]{0.31,0.60,0.02}{#1}}
\newcommand{\VariableTok}[1]{\textcolor[rgb]{0.00,0.00,0.00}{#1}}
\newcommand{\VerbatimStringTok}[1]{\textcolor[rgb]{0.31,0.60,0.02}{#1}}
\newcommand{\WarningTok}[1]{\textcolor[rgb]{0.56,0.35,0.01}{\textbf{\textit{#1}}}}
\usepackage{graphicx}
\makeatletter
\newsavebox\pandoc@box
\newcommand*\pandocbounded[1]{% scales image to fit in text height/width
  \sbox\pandoc@box{#1}%
  \Gscale@div\@tempa{\textheight}{\dimexpr\ht\pandoc@box+\dp\pandoc@box\relax}%
  \Gscale@div\@tempb{\linewidth}{\wd\pandoc@box}%
  \ifdim\@tempb\p@<\@tempa\p@\let\@tempa\@tempb\fi% select the smaller of both
  \ifdim\@tempa\p@<\p@\scalebox{\@tempa}{\usebox\pandoc@box}%
  \else\usebox{\pandoc@box}%
  \fi%
}
% Set default figure placement to htbp
\def\fps@figure{htbp}
\makeatother
\setlength{\emergencystretch}{3em} % prevent overfull lines
\providecommand{\tightlist}{%
  \setlength{\itemsep}{0pt}\setlength{\parskip}{0pt}}
\setcounter{secnumdepth}{5}
% definitions for citeproc citations
\NewDocumentCommand\citeproctext{}{}
\NewDocumentCommand\citeproc{mm}{%
  \begingroup\def\citeproctext{#2}\cite{#1}\endgroup}
\makeatletter
 % allow citations to break across lines
 \let\@cite@ofmt\@firstofone
 % avoid brackets around text for \cite:
 \def\@biblabel#1{}
 \def\@cite#1#2{{#1\if@tempswa , #2\fi}}
\makeatother
\newlength{\cslhangindent}
\setlength{\cslhangindent}{1.5em}
\newlength{\csllabelwidth}
\setlength{\csllabelwidth}{3em}
\newenvironment{CSLReferences}[2] % #1 hanging-indent, #2 entry-spacing
 {\begin{list}{}{%
  \setlength{\itemindent}{0pt}
  \setlength{\leftmargin}{0pt}
  \setlength{\parsep}{0pt}
  % turn on hanging indent if param 1 is 1
  \ifodd #1
   \setlength{\leftmargin}{\cslhangindent}
   \setlength{\itemindent}{-1\cslhangindent}
  \fi
  % set entry spacing
  \setlength{\itemsep}{#2\baselineskip}}}
 {\end{list}}
\usepackage{calc}
\newcommand{\CSLBlock}[1]{\hfill\break\parbox[t]{\linewidth}{\strut\ignorespaces#1\strut}}
\newcommand{\CSLLeftMargin}[1]{\parbox[t]{\csllabelwidth}{\strut#1\strut}}
\newcommand{\CSLRightInline}[1]{\parbox[t]{\linewidth - \csllabelwidth}{\strut#1\strut}}
\newcommand{\CSLIndent}[1]{\hspace{\cslhangindent}#1}
\usepackage{bookmark}
\IfFileExists{xurl.sty}{\usepackage{xurl}}{} % add URL line breaks if available
\urlstyle{same}
\hypersetup{
  pdftitle={LBRAI2219 - Modélisation des systèmes biologiques},
  pdfauthor={Maxime CORNEZ},
  hidelinks,
  pdfcreator={LaTeX via pandoc}}

\title{LBRAI2219 - Modélisation des systèmes biologiques}
\usepackage{etoolbox}
\makeatletter
\providecommand{\subtitle}[1]{% add subtitle to \maketitle
  \apptocmd{\@title}{\par {\large #1 \par}}{}{}
}
\makeatother
\subtitle{Modèle APSIM pour deux cultures}
\author{Maxime CORNEZ}
\date{2025-05-24}

\begin{document}
\maketitle

{
\setcounter{tocdepth}{2}
\tableofcontents
}
\section{Résumé}\label{ruxe9sumuxe9}

\begin{quote}
Résumé du travail, reprenant les résultats principaux et les
perspectives.
\end{quote}

\section{Introduction}\label{introduction}

\subsection{État de l'art}\label{uxe9tat-de-lart}

L'agriculture moderne est soumise à de nombreux défis à l'heure actuelle
: elle se doit de produire assez pour nourrir la population mondiale,
tout en respectant au mieux l'environnement et en s'adaptant aux
changements climatiques. Or, les terres disponibles pour l'agriculture
ne sont pas illimitées et la pression sur les ressources naturelles est
de plus en plus forte. De plus, les ressources en eau et en terre ne
sont pas distribuées de manière égale sur la planète, ce qui rend la
tâche encore plus complexe (Dudgeon et al. 2006).\\
L'agriculture dépend de ces deux ressources presque autant qu'elle
dépend du soleil, or avec l'augmentation de la population mondiale et
les changements climatiques, la disponibilité de l'eau et des terres
agricoles est mise à rude épreuve.\\
Aujourd'hui, si on combine les terres arables et les prairies,ces deux
biomes sont les plus présent à la surface de la Terre, occupant environ
40 \% de la surface terrestre. En outre, les terres agricoles sont
souvent utilisées de manière intensive, avec des pratiques telles que la
monoculture, l'utilisation excessive d'engrais et de pesticides, et la
déforestation pour créer de nouvelles terres agricoles. Ces pratiques
ont des conséquences néfastes sur l'environnement, notamment la perte de
biodiversité, la dégradation des sols et la pollution de l'eau (Foley et
al. 2005).

\hfill\break
La recherche agronomique s'est orientée vers des pratiques agricoles
plus durables, telles que l'agroécologie, l'agriculture de conservation
et la permaculture. Ces pratiques visent à réduire l'impact
environnemental de l'agriculture tout en maintenant des rendements
élevés. Elles reposent sur une meilleure compréhension des interactions
entre les plantes, les sols et les écosystèmes, ainsi que sur
l'utilisation de techniques de gestion intégrée des cultures (Bondeau et
al. 2007).

Pour s'adapter aux épisodes de stress hydrique et thermique, les plantes
déploient un éventail de réponses physiologiques coordonnées. D'une
part, elles ferment partiellement, voire totalement, leurs stomates pour
réduire les pertes d'eau, ajustent leur croissance, celle de leurs
racines et celle de leur feuillage, pour limiter l'évapotranspiration et
modifient l'orientation et la taille de leurs feuilles afin d'optimiser
la capture de la lumière sans souffrir de surchauffe. D'autre part,
elles optimisent l'efficacité de leur utilisation de l'eau par des
ajustements biochimiques, comme l'accumulation d'osmolytes, et par des
modifications de la composition de leur cuticule pour diminuer la perte
d'eau.\\
Au cœur de ces stratégies, l'architecture racinaire joue un rôle
décisif. Un enracinement profond permet d'accéder aux réserves hydriques
des horizons profonds, tandis qu'une croissance rapide des racines
favorise une exploration efficace du volume de sol. La capacité à
répartir finement les racines dans les différents niveaux du sol accroît
considérablement les probabilités de rencontrer des poches d'humidité.
De plus, certaines plantes développent une plasticité racinaire
importante, ajustant dynamiquement la ramification et l'élongation des
racines selon la disponibilité locale de l'eau, et renforcent leurs
interactions avec les champignons mycorhiziens pour améliorer
l'absorption hydrique et minérale. Ces traits racinaires combinés
forment un arsenal adaptatif essentiel pour maintenir la croissance et
la productivité en conditions de déficit hydrique(Munns 2002).

\hfill\break
Ce projet traitera plus spécifiquement de la permaculture et des
intercultures, techniques visant à maximiser l'espace disponible et les
ressources pour deux cultures ou plus occupant le même espace. Ces
techniques sont de plus en plus populaires dans les systèmes agricoles
durables, car elles permettent de diversifier les cultures, d'améliorer
la santé des sols et de réduire les besoins en intrants chimiques
(Brooker et al. 2015).

Le maïs (\emph{Zea mays L.}) est l'une des céréales les plus cultivées
au monde, environ 850 millions de tonnes produites par an sur
approximativement 162 millions d'hectares (Yara 2018). Son métabolisme
en C₄ lui garantit une photosynthèse très efficace et une bonne
tolérance aux fortes températures. Toutefois, il reste fortement
dépendant de l'eau, en particulier durant la floraison et la
pollinisation, lorsque le stress hydrique peut provoquer des pertes de
rendement importantes. Son réseau racinaire dense et profond, pouvant
aller jusqu'à 1m80, lui permet une captation efficace de l'eau et des
nutriments (Barnabás, Jäger, and Fehér 2008).

Il serait intéressant d'optimiser ces surfaces de culture en alliant les
plants de maïs avec d'autres plantes pouvant bénéficier de cet espace.
La sélection des bonnes variétés de plante entrainera un symbiose,
améliorant le rendement globale de production et permettant ainsi une
meilleure valorisation de l'eau et des ressources.\\
Les cultures le plus souvent associées au maïs sont souvent des espèces
de petite taille, profitant de l'ombrage apporté par les feuilles du
maïs ou de la structure de sa tige(Brooker et al. 2015; Dong et al.
2022; Nassary, Baijukya, and Ndakidemi 2020; Zhang and Li 2003). On
retrouve par exemple :

\begin{itemize}
\item
  des légumineuses comme les haricots (\emph{Phaseolus vulgaris}) : ils
  fixent l'azote atmosphérique dans le sol, améliorant ainsi la
  fertilité du sol et réduisant le besoin en engrais azotés pour le maïs
  et se servent des tiges de maïs comme support pour leurs tiges (Mudare
  et al. 2022).
\item
  des courges (\emph{Cucurbita pepo}) : elles ont de grandes feuilles
  qui fournissent de l'ombre au sol, réduisant l'évaporation de l'eau et
  maintenant une température du sol plus fraîche, ce qui est bénéfique
  pour le maïs. Elles aident également à contrôler les mauvaises herbes
  en couvrant le sol (Cryan et al. 2025).
\item
  des salades (\emph{Lactuca sativa}) : elles ont une croissance rapide
  et peuvent être récoltées avant que le maïs ne devienne trop grand,
  permettant ainsi une utilisation efficace de l'espace. Elles protègent
  le sol et exploitent la strate inférieure (Brennan 2013).
\end{itemize}

\subsection{Importance de l'approche par
modélisation}\label{importance-de-lapproche-par-moduxe9lisation}

Comme l'a dit Samuel P. Huntington ``Nous avons besoin de modèles
explicites ou implicites pour pouvoir ordonner et généraliser la
réalité, comprendre les relations causales entre les phénomènes,
anticiper et, si nous avons de la chance, prédire les développements
futurs, distinguer l'essentiel de l'accessoire et nous indiquer les
chemins à suivre pour atteindre nos objectifs''. Un modèle n'est jamais
une parfaite représentation de la réalité, mais il permet de simplifier
et d'organiser les connaissances pour mieux comprendre les systèmes
complexes. En agronomie, la modélisation est un outil essentiel pour
simuler le comportement des cultures et des systèmes agricoles, en
tenant compte des interactions entre les plantes, le sol et
l'environnement, sans devoir nécessairement passer par une étape
expérimentale longue et coûteuse (Tremblay and Wallach 2004).

Le modèle APSIM (pour Agricultural Production Systems sIMulator) est un
modèle de simulation de culture qui permet de simuler la croissance des
cultures, la dynamique du sol et les interactions entre les plantes et
l'environnement. Il a été développé en 1995 par une équipe
multiuniversitaire et est encore utilisé et amélioré à ce jour. APSIM
est un modèle modulaire, ce qui signifie qu'il peut être adapté à
différents systèmes agricoles et à différentes conditions
environnementales. Il est capable de simuler la croissance des cultures
en tenant compte des facteurs climatiques, hydriques et nutritionnels,
ainsi que des interactions entre les plantes et le sol (McCown et al.
1996; Keating et al. 2003).\\

\section{Matériel et méthodes}\label{matuxe9riel-et-muxe9thodes}

\subsection{Modèles}\label{moduxe8les}

L'étude repose sur une modélisation d'APSIM en combinant deux cultures
(maïs et autre) en interculture. Le modèle APSIM est utilisé pour
simuler la croissance des cultures, la dynamique du sol et les
interactions entre les plantes et l'environnement.

\begin{itemize}
\item
  \textbf{APSIM mono} : Ce modèle a pour but de simuler la production de
  biomasse chez le maïs en prenant en compte à la fois les contraintes
  hydriques et lumineuses. Il s'appuie d'une part sur les équations de
  Monteith, qui relient la croissance à la quantité de rayonnement
  intercepté, et d'autre part sur le concept d'efficience de
  transpiration, qui traduit la conversion de l'eau prélevée en biomassе
  sous stress hydrique. Ensemble, ces deux approches permettent de
  quantifier l'impact combiné de la lumière et de la disponibilité en
  eau sur le développement de la culture. (\textbf{draye?}; Hammer
  2009)\\
  La version présentée ici a été légèrement remise en forme par Alice
  Falzon et moi-même afin d'en améliorer l'érgonomie et la facilité
  d'utilisation.
\item
  \textbf{APSIM duo} : Ce modèle est une extension du modèle mono, qui
  permet de simuler la production de biomasse pour deux cultures en
  interculture. Il prend en compte les interactions entre les deux
  cultures pour l'eau et la profondeur des racines. Le modèle est
  capable de simuler la croissance des deux cultures en tenant compte
  des facteurs climatiques et hydriques, ainsi que des interactions
  entre les plantes et le sol. Le modèle inclut un ``arbitre''
  déterminant quelle culture recevra prioritairement l'eau en cas de
  déficit hydrique, cet arbitre est inclu directement dans la fonction
  du modèle.\\
  Par simplicité, l'hypothèse que les deux cultures n'ont pas
  d'intéractions au niveau de la lumière a été faite, ce qui signifie
  que la lumière est répartie entre les deux cultures selon la
  disponibilité.\\
  La conversion de APSIM mono vers duo a représenté la plus grande
  partie de ce projet.
\end{itemize}

Le plan de modélisation est structuré comme suit :

\begin{enumerate}
\def\labelenumi{\arabic{enumi}.}
\item
  Obtention des paramètres de cultures pour chaque culture désirée. Puis
  des paramètres du sol et de l'expansion foliaire
\item
  Obtention des paramètres météo, soit via un jeu de données généré
  artificiellement, soit via un jeu de données réelles obtenu via
  Open-Meteo (\textbf{open-meteo\_\_nodate?}) ; sélection de la source
  via le switch installé dans le code.
\item
  Fixation des densités de culture.
\item
  Simulation de la production de biomasse avec APSIM.

  \begin{enumerate}
  \def\labelenumii{\arabic{enumii}.}
  \item
    Pour le maïs seul avec Apsim mono.
  \item
    En combinant le maïs avec une autre culture avec Apsim duo.
  \end{enumerate}
\item
  Analyse et comparaison des résultats.
\end{enumerate}

\subsection{Packages}\label{packages}

\begin{itemize}
\item
  \textbf{Importation et manipulation de données}

  \begin{itemize}
  \item
    \emph{readxl} : permet de lire directement des fichiers Excel (.xls,
    .xlsx) en important les feuilles de calcul sous forme de data
    frames.
  \item
    \emph{dplyr} et \emph{tidyr} : offrent un ensemble de fonctions
    (p.~ex. filter(), select(), mutate() ) pour nettoyer, transformer et
    reformater les données de façon claire et fluide.
  \end{itemize}
\item
  \textbf{Visualisation}

  \begin{itemize}
  \tightlist
  \item
    \emph{ggplot2} : propose un système de grammaire graphique puissant
    pour créer des graphiques élégants et personnalisables
    (histogrammes, nuages de points, séries temporelles, etc.) à partir
    de de data frames.
  \end{itemize}
\item
  \textbf{Accès et traitement de données web}

  \begin{itemize}
  \item
    \emph{httr} : facilite les requêtes HTTP (GET, POST,
    authentification) pour interagir avec des API ou récupérer des
    ressources distantes.
  \item
    \emph{jsonlite} : convertit des fichiers JSON en objets R (et
    inversement), ce qui rend l'exploitation des réponses d'API et le
    stockage des données JSON plus accessibles.
  \end{itemize}
\end{itemize}

\begin{Shaded}
\begin{Highlighting}[]
\CommentTok{\# Packages requis}
\FunctionTok{library}\NormalTok{(readxl)}
\FunctionTok{library}\NormalTok{(dplyr)}
\FunctionTok{library}\NormalTok{(tidyr)}
\FunctionTok{library}\NormalTok{(ggplot2)}
\FunctionTok{library}\NormalTok{(jsonlite)}
\FunctionTok{library}\NormalTok{(httr)}
\end{Highlighting}
\end{Shaded}

\subsection{Données d'entrée}\label{donnuxe9es-dentruxe9e}

\subsubsection{Paramètres des
cultures}\label{paramuxe8tres-des-cultures}

Dans la mesure du possible, les paramètres ont été extraits de la
littérature. Lorsque certaines données faisaient défaut, des
extrapolations ont été réalisées à partir des connaissances acquises au
cours de ma formation, dans un but de différenciation de l'espèce par
rapport au maïs. Seules les données relatives au maïs peuvent être
considérées comme pleinement fiables, puisqu'elles proviennent d'un
exercice pratique donné par le professeur Draye Rouphael and Colla
(2005).

\begin{Shaded}
\begin{Highlighting}[]
\NormalTok{mais }\OtherTok{\textless{}{-}} \FunctionTok{list}\NormalTok{(}
  \AttributeTok{RUE               =} \FloatTok{1.6}\NormalTok{,   }\CommentTok{\# Radiation Use Efficiency [g/MJ]}
  \AttributeTok{TEc               =} \DecValTok{9}\NormalTok{,     }\CommentTok{\# Coefficient d\textquotesingle{}efficience de la transpiration [Pa]}
  \AttributeTok{VPR               =} \DecValTok{20}\NormalTok{,    }\CommentTok{\# Vitesse production de racines [mm/jour]}
  \AttributeTok{CroissPotLAI      =} \FloatTok{0.1}\NormalTok{,   }\CommentTok{\# Croissance potentielle du LAI }
  \AttributeTok{k                 =} \FloatTok{0.45}\NormalTok{,  }\CommentTok{\# Coefficient d\textquotesingle{}extinction de la lumière}
  \AttributeTok{LAI\_initial       =} \FloatTok{1.5}\NormalTok{,   }
  \AttributeTok{Biomasse\_initiale =} \DecValTok{45}     
\NormalTok{)}

\NormalTok{sorgho }\OtherTok{\textless{}{-}} \FunctionTok{list}\NormalTok{(}
  \AttributeTok{RUE               =} \FloatTok{1.25}\NormalTok{,  }\CommentTok{\# Radiation Use Efficiency [g/MJ]}
  \AttributeTok{TEc               =} \DecValTok{9}\NormalTok{,     }\CommentTok{\# Coefficient d\textquotesingle{}efficience de la transpiration [Pa]}
  \AttributeTok{VPR               =} \DecValTok{20}\NormalTok{,    }\CommentTok{\# Vitesse production de racines [mm/jour]}
  \AttributeTok{CroissPotLAI      =} \FloatTok{0.1}\NormalTok{,   }\CommentTok{\# Croissance potentielle du LAI }
  \AttributeTok{k                 =} \FloatTok{0.45}\NormalTok{,  }\CommentTok{\# Coefficient d\textquotesingle{}extinction de la lumière}
  \AttributeTok{LAI\_initial       =} \FloatTok{1.5}\NormalTok{,   }
  \AttributeTok{Biomasse\_initiale =} \DecValTok{45}     
\NormalTok{)}

\NormalTok{haricot }\OtherTok{\textless{}{-}} \FunctionTok{list}\NormalTok{(}
  \AttributeTok{RUE               =} \FloatTok{1.78}\NormalTok{,  }\CommentTok{\# Radiation Use Efficiency [g/MJ]}
  \AttributeTok{TEc               =} \FloatTok{4.9}\NormalTok{,   }\CommentTok{\# Coefficient d\textquotesingle{}efficience de la transpiration [Pa]}
  \AttributeTok{VPR               =} \FloatTok{13.5}\NormalTok{,  }\CommentTok{\# Vitesse production de racines [mm/jour]}
  \AttributeTok{CroissPotLAI      =} \FloatTok{0.1}\NormalTok{,   }\CommentTok{\# Croissance potentielle du LAI }
  \AttributeTok{k                 =} \FloatTok{0.6}\NormalTok{,   }\CommentTok{\# Coefficient d\textquotesingle{}extinction de la lumière}
  \AttributeTok{LAI\_initial       =} \FloatTok{1.0}\NormalTok{,   }
  \AttributeTok{Biomasse\_initiale =} \DecValTok{20}     
\NormalTok{)}

\NormalTok{courge }\OtherTok{\textless{}{-}} \FunctionTok{list}\NormalTok{(}
  \AttributeTok{RUE               =} \FloatTok{4.1}\NormalTok{,   }\CommentTok{\# Radiation Use Efficiency [g/MJ]}
  \AttributeTok{TEc               =} \FloatTok{2.9}\NormalTok{,   }\CommentTok{\# Coefficient d\textquotesingle{}efficience de la transpiration [Pa]}
  \AttributeTok{VPR               =} \DecValTok{15}\NormalTok{,    }\CommentTok{\# Vitesse production de racines [mm/jour]}
  \AttributeTok{CroissPotLAI      =} \FloatTok{0.12}\NormalTok{,  }\CommentTok{\# Croissance potentielle du LAI }
  \AttributeTok{k                 =} \FloatTok{0.72}\NormalTok{,  }\CommentTok{\# Coefficient d\textquotesingle{}extinction de la lumière}
  \AttributeTok{LAI\_initial       =} \FloatTok{0.8}\NormalTok{,   }
  \AttributeTok{Biomasse\_initiale =} \DecValTok{35}
\NormalTok{)}


\NormalTok{salade }\OtherTok{\textless{}{-}} \FunctionTok{list}\NormalTok{(}
  \AttributeTok{RUE               =} \FloatTok{1.2}\NormalTok{,   }\CommentTok{\# Radiation Use Efficiency [g/MJ]}
  \AttributeTok{TEc               =} \DecValTok{3}\NormalTok{,     }\CommentTok{\# Coefficient d\textquotesingle{}efficience de la transpiration [Pa]}
  \AttributeTok{VPR               =} \DecValTok{8}\NormalTok{,     }\CommentTok{\# Vitesse production de racines [mm/jour]}
  \AttributeTok{CroissPotLAI      =} \FloatTok{0.15}\NormalTok{,  }\CommentTok{\# Croissance potentielle du LAI }
  \AttributeTok{k                 =} \FloatTok{0.5}\NormalTok{,   }\CommentTok{\# Coefficient d\textquotesingle{}extinction de la lumière}
  \AttributeTok{LAI\_initial       =} \FloatTok{0.6}\NormalTok{,   }
  \AttributeTok{Biomasse\_initiale =} \DecValTok{15}     
\NormalTok{)}
\end{Highlighting}
\end{Shaded}

\subsubsection{Paramètres du sol}\label{paramuxe8tres-du-sol}

Paramètres générés artificiellement pour avoir un sol d' 1m d'épaisseur
contenant un minimum de 40mm d'eau et un maximum de 100mm d'eau.\\
L'hyphothèse que le kl est une valeur propre au sol a été posée pour
simplifier les calculs. En réalité, le kl est variable selon de nombreux
critères propres au sol et aux plantes occupant le sol.

\begin{Shaded}
\begin{Highlighting}[]
\NormalTok{soil\_params }\OtherTok{\textless{}{-}} \FunctionTok{data.frame}\NormalTok{(}
  \AttributeTok{Horizon     =} \FunctionTok{c}\NormalTok{(}\DecValTok{1}\NormalTok{, }\DecValTok{2}\NormalTok{, }\DecValTok{3}\NormalTok{),}
  \AttributeTok{Epaisseur   =} \FunctionTok{c}\NormalTok{(}\DecValTok{300}\NormalTok{, }\DecValTok{300}\NormalTok{, }\DecValTok{400}\NormalTok{),   }\CommentTok{\# mm}
  \AttributeTok{li          =} \FunctionTok{c}\NormalTok{(}\DecValTok{40}\NormalTok{,  }\DecValTok{40}\NormalTok{,  }\DecValTok{40}\NormalTok{),    }\CommentTok{\# Limite inférieure d\textquotesingle{}eau}
  \AttributeTok{ls          =} \FunctionTok{c}\NormalTok{(}\DecValTok{100}\NormalTok{, }\DecValTok{100}\NormalTok{, }\DecValTok{100}\NormalTok{),   }\CommentTok{\# Limite supérieure d\textquotesingle{}eau}
  \AttributeTok{es          =} \FunctionTok{c}\NormalTok{(}\DecValTok{100}\NormalTok{, }\DecValTok{100}\NormalTok{, }\DecValTok{100}\NormalTok{),   }\CommentTok{\# Eau disponible dans sol (= sw)}
  \AttributeTok{es\_h        =} \FunctionTok{c}\NormalTok{(}\DecValTok{40}\NormalTok{, }\DecValTok{30}\NormalTok{, }\DecValTok{30}\NormalTok{),      }\CommentTok{\# Eau disponible dans sol par horizon ; es\_h1+es\_h2+es\_h3 = es}
  \AttributeTok{kl          =} \FunctionTok{c}\NormalTok{(}\FloatTok{0.06}\NormalTok{, }\FloatTok{0.05}\NormalTok{, }\FloatTok{0.05}\NormalTok{) }\CommentTok{\# Taux d\textquotesingle{}extraction [mm/jour]}
\NormalTok{)}
\end{Highlighting}
\end{Shaded}

\subsubsection{Expansion foliaire}\label{expansion-foliaire}

\begin{Shaded}
\begin{Highlighting}[]
\CommentTok{\# Culture 1}
\NormalTok{expansion\_foliaire }\OtherTok{\textless{}{-}} \FunctionTok{data.frame}\NormalTok{(}
  \AttributeTok{OD  =} \FunctionTok{c}\NormalTok{(}\FloatTok{0.5}\NormalTok{, }\FloatTok{1.5}\NormalTok{, }\DecValTok{4}\NormalTok{), }\CommentTok{\# Offre sur demande}
  \AttributeTok{CEF =} \FunctionTok{c}\NormalTok{(}\DecValTok{0}\NormalTok{,   }\DecValTok{1}\NormalTok{,   }\DecValTok{1}\NormalTok{)  }\CommentTok{\# Coefficient d\textquotesingle{}expansion foliaire}
\NormalTok{)}

\CommentTok{\# Culture 2}
\NormalTok{expansion\_foliaire2 }\OtherTok{\textless{}{-}} \FunctionTok{data.frame}\NormalTok{(}
  \AttributeTok{OD  =} \FunctionTok{c}\NormalTok{(}\FloatTok{0.4}\NormalTok{, }\FloatTok{1.0}\NormalTok{, }\FloatTok{3.5}\NormalTok{), }\CommentTok{\# Offre sur demande}
  \AttributeTok{CEF =} \FunctionTok{c}\NormalTok{(}\DecValTok{0}\NormalTok{,   }\FloatTok{0.8}\NormalTok{, }\DecValTok{1}\NormalTok{)    }\CommentTok{\# Coefficient d\textquotesingle{}expansion foliaire}

\NormalTok{)}
\end{Highlighting}
\end{Shaded}

\subsubsection{Paramètres météo}\label{paramuxe8tres-muxe9tuxe9o}

\begin{Shaded}
\begin{Highlighting}[]
\CommentTok{\# Données météo réelles }
\NormalTok{lat }\OtherTok{\textless{}{-}} \FloatTok{50.666265}\NormalTok{; lon }\OtherTok{\textless{}{-}} \FloatTok{4.622322}   \CommentTok{\# Localisation de la zone d\textquotesingle{}étude : 50.666265 ; 4.622322 = batiment De serres}
\NormalTok{start\_date }\OtherTok{\textless{}{-}} \StringTok{"2024{-}06{-}01"}\NormalTok{; end\_date }\OtherTok{\textless{}{-}} \StringTok{"2024{-}06{-}30"}           \CommentTok{\# Période de simulation}
\NormalTok{res }\OtherTok{\textless{}{-}} \FunctionTok{GET}\NormalTok{(}\StringTok{"https://archive{-}api.open{-}meteo.com/v1/archive"}\NormalTok{,    }\CommentTok{\# Obtention des données de la station météo belge}
           \AttributeTok{query =} \FunctionTok{list}\NormalTok{(}\AttributeTok{latitude =}\NormalTok{ lat, }\AttributeTok{longitude =}\NormalTok{ lon,}
                        \AttributeTok{start\_date =}\NormalTok{ start\_date, }\AttributeTok{end\_date =}\NormalTok{ end\_date,}
                        \AttributeTok{daily =} \StringTok{"temperature\_2m\_max,temperature\_2m\_min,shortwave\_radiation\_sum"}\NormalTok{,}
                        \AttributeTok{timezone =} \StringTok{"Europe/Brussels"}\NormalTok{))}
\NormalTok{meteo\_data }\OtherTok{\textless{}{-}} \FunctionTok{fromJSON}\NormalTok{(}\FunctionTok{content}\NormalTok{(res, }\StringTok{"text"}\NormalTok{))}\SpecialCharTok{$}\NormalTok{daily             }\CommentTok{\# Importation du fichier de données JSON}

\NormalTok{meteo\_df }\OtherTok{\textless{}{-}} \FunctionTok{as.data.frame}\NormalTok{(meteo\_data) }\SpecialCharTok{\%\textgreater{}\%}
  \FunctionTok{rename}\NormalTok{(}\AttributeTok{Date =}\NormalTok{ time, }\AttributeTok{Tmax =}\NormalTok{ temperature\_2m\_max,}
         \AttributeTok{Tmin =}\NormalTok{ temperature\_2m\_min, }\AttributeTok{Radiation =}\NormalTok{ shortwave\_radiation\_sum) }\SpecialCharTok{\%\textgreater{}\%}
  
  \FunctionTok{mutate}\NormalTok{(}\AttributeTok{Jours =} \DecValTok{1}\SpecialCharTok{:}\FunctionTok{n}\NormalTok{(),  }\CommentTok{\# fonctions pour calculer vdp                                    }
         \AttributeTok{svpTmax =} \FloatTok{6.1078} \SpecialCharTok{*} \FunctionTok{exp}\NormalTok{(}\FloatTok{17.269} \SpecialCharTok{*}\NormalTok{ Tmax }\SpecialCharTok{/}\NormalTok{ (}\FloatTok{237.3} \SpecialCharTok{+}\NormalTok{ Tmax)) }\SpecialCharTok{*} \FloatTok{0.10}\NormalTok{,}
         \AttributeTok{svpTmin =} \FloatTok{6.1078} \SpecialCharTok{*} \FunctionTok{exp}\NormalTok{(}\FloatTok{17.269} \SpecialCharTok{*}\NormalTok{ Tmin }\SpecialCharTok{/}\NormalTok{ (}\FloatTok{237.3} \SpecialCharTok{+}\NormalTok{ Tmin)) }\SpecialCharTok{*} \FloatTok{0.10}\NormalTok{,}
         \AttributeTok{VPDcalc =} \FloatTok{0.75} \SpecialCharTok{*}\NormalTok{ (svpTmax }\SpecialCharTok{{-}}\NormalTok{ svpTmin) }\SpecialCharTok{*} \DecValTok{10}\NormalTok{)}

\CommentTok{\# Génération de données artificielles}

\NormalTok{facteur\_externe }\OtherTok{\textless{}{-}} \FunctionTok{data.frame}\NormalTok{(}
  \CommentTok{\# Durée de la simulation (jours)}
  \AttributeTok{Jours =} \DecValTok{30}\SpecialCharTok{:}\DecValTok{60}\NormalTok{, }
  
  \CommentTok{\# Radiation (MJ/m2)}
  \AttributeTok{Radiation =} \FunctionTok{c}\NormalTok{( }
    \DecValTok{27}\NormalTok{, }\DecValTok{27}\NormalTok{, }\DecValTok{14}\NormalTok{, }\DecValTok{24}\NormalTok{, }\DecValTok{23}\NormalTok{, }\DecValTok{21}\NormalTok{, }
    \DecValTok{23}\NormalTok{, }\DecValTok{25}\NormalTok{, }\DecValTok{17}\NormalTok{, }\DecValTok{14}\NormalTok{, }\DecValTok{26}\NormalTok{, }\DecValTok{26}\NormalTok{, }
    \DecValTok{10}\NormalTok{, }\DecValTok{26}\NormalTok{, }\DecValTok{30}\NormalTok{, }\DecValTok{27}\NormalTok{, }\DecValTok{27}\NormalTok{, }\DecValTok{29}\NormalTok{, }
    \DecValTok{27}\NormalTok{, }\DecValTok{26}\NormalTok{, }\DecValTok{25}\NormalTok{, }\DecValTok{23}\NormalTok{, }\DecValTok{14}\NormalTok{, }\DecValTok{25}\NormalTok{, }
    \DecValTok{22}\NormalTok{, }\DecValTok{20}\NormalTok{, }\DecValTok{22}\NormalTok{, }\DecValTok{24}\NormalTok{, }\DecValTok{28}\NormalTok{, }\DecValTok{25}\NormalTok{, }\DecValTok{25}
\NormalTok{  ),}
  
  \CommentTok{\# Tmax (°C)}
  \AttributeTok{Tmax =} \FunctionTok{c}\NormalTok{( }
    \FloatTok{32.3}\NormalTok{, }\FloatTok{31.0}\NormalTok{, }\FloatTok{26.6}\NormalTok{, }\FloatTok{26.0}\NormalTok{, }\FloatTok{26.6}\NormalTok{, }\FloatTok{29.5}\NormalTok{, }
    \FloatTok{30.8}\NormalTok{, }\FloatTok{32.5}\NormalTok{, }\FloatTok{32.3}\NormalTok{, }\FloatTok{25.2}\NormalTok{, }\FloatTok{27.8}\NormalTok{, }\FloatTok{28.2}\NormalTok{, }
    \FloatTok{27.3}\NormalTok{, }\FloatTok{28.6}\NormalTok{, }\FloatTok{28.6}\NormalTok{, }\FloatTok{28.3}\NormalTok{, }\FloatTok{27.6}\NormalTok{, }\FloatTok{31.0}\NormalTok{, }
    \FloatTok{35.0}\NormalTok{, }\FloatTok{34.3}\NormalTok{, }\FloatTok{31.2}\NormalTok{, }\FloatTok{32.7}\NormalTok{, }\FloatTok{29.9}\NormalTok{, }\FloatTok{30.8}\NormalTok{, }
    \FloatTok{31.2}\NormalTok{, }\FloatTok{28.4}\NormalTok{, }\FloatTok{27.7}\NormalTok{, }\FloatTok{31.4}\NormalTok{, }\FloatTok{33.0}\NormalTok{, }\FloatTok{33.2}\NormalTok{, }\FloatTok{32.7}
\NormalTok{  ),}
  
  \CommentTok{\# Tmin (°C)}
  \AttributeTok{Tmin =} \FunctionTok{c}\NormalTok{( }
    \FloatTok{16.4}\NormalTok{, }\FloatTok{15.8}\NormalTok{, }\FloatTok{15.6}\NormalTok{, }\FloatTok{10.0}\NormalTok{, }\FloatTok{11.7}\NormalTok{, }\FloatTok{13.0}\NormalTok{, }
    \FloatTok{16.5}\NormalTok{, }\FloatTok{13.8}\NormalTok{, }\FloatTok{16.7}\NormalTok{, }\FloatTok{16.7}\NormalTok{, }\FloatTok{15.8}\NormalTok{, }\FloatTok{12.8}\NormalTok{, }
    \FloatTok{17.3}\NormalTok{, }\FloatTok{11.3}\NormalTok{, }\FloatTok{13.7}\NormalTok{, }\FloatTok{13.4}\NormalTok{, }\FloatTok{13.7}\NormalTok{, }\FloatTok{12.2}\NormalTok{, }
    \FloatTok{14.7}\NormalTok{, }\FloatTok{20.4}\NormalTok{, }\FloatTok{15.5}\NormalTok{, }\FloatTok{17.9}\NormalTok{, }\FloatTok{18.4}\NormalTok{, }\FloatTok{16.0}\NormalTok{, }
    \FloatTok{16.1}\NormalTok{, }\FloatTok{18.0}\NormalTok{, }\FloatTok{14.9}\NormalTok{, }\FloatTok{16.0}\NormalTok{, }\FloatTok{16.2}\NormalTok{, }\FloatTok{17.2}\NormalTok{, }\FloatTok{18.3}
\NormalTok{  ),}
  
  \CommentTok{\# VPDobs (hPa)}
  \AttributeTok{VPDobs =} \FunctionTok{c}\NormalTok{( }
    \DecValTok{19}\NormalTok{, }\DecValTok{17}\NormalTok{, }\DecValTok{16}\NormalTok{, }\DecValTok{14}\NormalTok{, }\DecValTok{12}\NormalTok{, }\DecValTok{15}\NormalTok{, }
    \DecValTok{18}\NormalTok{, }\DecValTok{19}\NormalTok{, }\DecValTok{23}\NormalTok{, }\DecValTok{21}\NormalTok{, }\DecValTok{15}\NormalTok{, }\DecValTok{14}\NormalTok{, }
    \DecValTok{20}\NormalTok{, }\DecValTok{16}\NormalTok{, }\DecValTok{17}\NormalTok{, }\DecValTok{17}\NormalTok{, }\DecValTok{15}\NormalTok{, }\DecValTok{15}\NormalTok{, }
    \DecValTok{16}\NormalTok{, }\DecValTok{17}\NormalTok{, }\DecValTok{18}\NormalTok{, }\DecValTok{21}\NormalTok{, }\DecValTok{22}\NormalTok{, }\DecValTok{17}\NormalTok{, }
    \DecValTok{18}\NormalTok{, }\DecValTok{20}\NormalTok{, }\DecValTok{18}\NormalTok{, }\DecValTok{19}\NormalTok{, }\DecValTok{19}\NormalTok{, }\DecValTok{20}\NormalTok{, }\DecValTok{21}
\NormalTok{  )}
\NormalTok{)}

\CommentTok{\# Fonction pour calculer saturated vapour pressure (SVP)}
\NormalTok{svp }\OtherTok{\textless{}{-}} \ControlFlowTok{function}\NormalTok{(T) \{ }\CommentTok{\# satured vapour pressure [kPa]}
  \FloatTok{6.1078} \SpecialCharTok{*} \FunctionTok{exp}\NormalTok{(}\FloatTok{17.269} \SpecialCharTok{*}\NormalTok{ T }\SpecialCharTok{/}\NormalTok{ (}\FloatTok{237.3} \SpecialCharTok{+}\NormalTok{ T)) }\SpecialCharTok{*} \FloatTok{0.10}
\NormalTok{\}}

\CommentTok{\# Application à Tmax}
\NormalTok{facteur\_externe}\SpecialCharTok{$}\NormalTok{svpTmax }\OtherTok{\textless{}{-}} \FunctionTok{svp}\NormalTok{(facteur\_externe}\SpecialCharTok{$}\NormalTok{Tmax) }\CommentTok{\# pression saturante à Tmax}

\CommentTok{\# Application à Tmin}
\NormalTok{facteur\_externe}\SpecialCharTok{$}\NormalTok{svpTmin }\OtherTok{\textless{}{-}} \FunctionTok{svp}\NormalTok{(facteur\_externe}\SpecialCharTok{$}\NormalTok{Tmin) }\CommentTok{\# pression saturante à Tmin}

\NormalTok{VPDfrac }\OtherTok{\textless{}{-}} \FloatTok{0.75} \CommentTok{\# par défaut on prend 0.75}


\CommentTok{\# Fonction pour calculer VPDcalc}
\NormalTok{calc\_VPDcalc }\OtherTok{\textless{}{-}} \ControlFlowTok{function}\NormalTok{(svpTmax, svpTmin, VPDfrac) \{}
\NormalTok{  VPDfrac }\SpecialCharTok{*}\NormalTok{ (svpTmax }\SpecialCharTok{{-}}\NormalTok{ svpTmin)}\SpecialCharTok{*}\DecValTok{10}
\NormalTok{\}}

\CommentTok{\# Ajout de la colonne VPDcalc}
\NormalTok{facteur\_externe}\SpecialCharTok{$}\NormalTok{VPDcalc }\OtherTok{\textless{}{-}} \FunctionTok{calc\_VPDcalc}\NormalTok{(}
  \AttributeTok{svpTmax  =}\NormalTok{ facteur\_externe}\SpecialCharTok{$}\NormalTok{svpTmax,}
  \AttributeTok{svpTmin  =}\NormalTok{ facteur\_externe}\SpecialCharTok{$}\NormalTok{svpTmin,}
  \AttributeTok{VPDfrac  =}\NormalTok{ VPDfrac}
\NormalTok{)}

\DocumentationTok{\#\#\#\#\#\#\#\#\#\#\#\#\#\#\#\#\#\#\#\#\#\#\#\#\#\#\#\#\#\#\#\#\#\#\#\#\#\#\#\#\#\#\#\#\#\#\#\#\#\#\#\#\#\#\#\#\#\#\#\#\#\#\#\#\#\#\#\#\#\#\#\#\#\#\#\#\#\#\#}
\CommentTok{\# Choix de la source de données : "reel" ou "artificiel" (switch)}
\DocumentationTok{\#\#\#\#\#\#\#\#\#\#\#\#\#\#\#\#\#\#\#\#\#\#\#\#\#\#\#\#\#\#\#\#\#\#\#\#\#\#\#\#\#\#\#\#\#\#\#\#\#\#\#\#\#\#\#\#\#\#\#\#\#\#\#\#\#\#\#\#\#\#\#\#\#\#\#\#\#\#\#}
\NormalTok{data\_source }\OtherTok{\textless{}{-}} \StringTok{"artificiel"}    \CommentTok{\# mettre "artificiel" si on veux le jeu généré}

\ControlFlowTok{if}\NormalTok{ (data\_source }\SpecialCharTok{==} \StringTok{"reel"}\NormalTok{) \{}
\NormalTok{  facteur\_externe }\OtherTok{\textless{}{-}}\NormalTok{ meteo\_df }\SpecialCharTok{\%\textgreater{}\%}
    \FunctionTok{select}\NormalTok{(Jours, Radiation, Tmax, Tmin, VPDcalc)}
\NormalTok{\} }\ControlFlowTok{else} \ControlFlowTok{if}\NormalTok{ (data\_source }\SpecialCharTok{==} \StringTok{"artificiel"}\NormalTok{) \{}
\NormalTok{  facteur\_externe }\OtherTok{\textless{}{-}}\NormalTok{ facteur\_externe }\SpecialCharTok{\%\textgreater{}\%}
    \FunctionTok{select}\NormalTok{(Jours, Radiation, Tmax, Tmin, VPDcalc)}
\NormalTok{\} }\ControlFlowTok{else}\NormalTok{ \{}
  \FunctionTok{stop}\NormalTok{(}\StringTok{"data\_source doit être \textquotesingle{}reel\textquotesingle{} ou \textquotesingle{}artificiel\textquotesingle{}"}\NormalTok{)}
\NormalTok{\}}
\end{Highlighting}
\end{Shaded}

\subsubsection{Densité des deux
cultures}\label{densituxe9-des-deux-cultures}

\begin{Shaded}
\begin{Highlighting}[]
\NormalTok{Densite1 }\OtherTok{\textless{}{-}} \FloatTok{0.5} \CommentTok{\# Densité de la culture 1 }
\NormalTok{Densite2 }\OtherTok{\textless{}{-}} \DecValTok{1}\SpecialCharTok{{-}}\NormalTok{Densite1 }\CommentTok{\# Densité de la culture 2 }
\end{Highlighting}
\end{Shaded}

\subsection{Modèles utilisés}\label{moduxe8les-utilisuxe9s}

Importation des deux modèles utilisés.

\begin{Shaded}
\begin{Highlighting}[]
\FunctionTok{source}\NormalTok{(}\StringTok{"apsim\_mono.R"}\NormalTok{)  }\CommentTok{\# Apsim monoculture}
\FunctionTok{source}\NormalTok{(}\StringTok{"apsim\_duo.R"}\NormalTok{)   }\CommentTok{\# Apsim double culture}
\end{Highlighting}
\end{Shaded}

\section{Résultats}\label{ruxe9sultats}

\subsection{Simulation des cultures}\label{simulation-des-cultures}

\begin{Shaded}
\begin{Highlighting}[]
\NormalTok{resultat\_mais }\OtherTok{\textless{}{-}} \FunctionTok{simulate}\NormalTok{(facteur\_externe, soil\_params, mais, expansion\_foliaire) }\CommentTok{\# simulation maïs seul}

\NormalTok{resultats\_mais\_haricot }\OtherTok{\textless{}{-}} \FunctionTok{simulate\_two}\NormalTok{(facteur\_externe, soil\_params, mais, haricot, expansion\_foliaire, expansion\_foliaire2, Densite1, Densite2) }\CommentTok{\# simulation maïs + haricot}

\NormalTok{resultats\_mais\_courge }\OtherTok{\textless{}{-}} \FunctionTok{simulate\_two}\NormalTok{(facteur\_externe, soil\_params, mais, courge, expansion\_foliaire, expansion\_foliaire2, Densite1, Densite2) }\CommentTok{\# simulation maïs + courge}

\NormalTok{resultats\_mais\_salade }\OtherTok{\textless{}{-}} \FunctionTok{simulate\_two}\NormalTok{(facteur\_externe, soil\_params, mais, salade, expansion\_foliaire, expansion\_foliaire2, Densite1, Densite2) }\CommentTok{\# simulation maïs + salade}
\end{Highlighting}
\end{Shaded}

\subsection{Modélisation de l'évolution de la culture (bah
oui)}\label{moduxe9lisation-de-luxe9volution-de-la-culture-bah-oui}

Lorem ipsum dolor sit amet, consectetur adipiscing elit, sed do eiusmod
tempor incididunt ut labore et dolore magna aliqua. Ut enim ad minim
veniam, quis nostrud exercitation ullamco laboris nisi ut aliquip ex ea
commodo consequat. Duis aute irure dolor in reprehenderit in voluptate
velit esse cillum dolore eu fugiat nulla pariatur. Excepteur sint
occaecat cupidatat non proident, sunt in culpa qui officia deserunt
mollit anim id est laborum.

\subsection{Couplage entre le modèle racinaire et le modèle de culture
(remplacer par couplage de 2
cultures)}\label{couplage-entre-le-moduxe8le-racinaire-et-le-moduxe8le-de-culture-remplacer-par-couplage-de-2-cultures}

Lorem ipsum dolor sit amet, consectetur adipiscing elit, sed do eiusmod
tempor incididunt ut labore et dolore magna aliqua. Ut enim ad minim
veniam, quis nostrud exercitation ullamco laboris nisi ut aliquip ex ea
commodo consequat. Duis aute irure dolor in reprehenderit in voluptate
velit esse cillum dolore eu fugiat nulla pariatur. Excepteur sint
occaecat cupidatat non proident, sunt in culpa qui officia deserunt
mollit anim id est laborum.

\subsection{Analyse de l'interaction entre rendements et racines (kl
?)}\label{analyse-de-linteraction-entre-rendements-et-racines-kl}

Lorem ipsum dolor sit amet, consectetur adipiscing elit, sed do eiusmod
tempor incididunt ut labore et dolore magna aliqua. Ut enim ad minim
veniam, quis nostrud exercitation ullamco laboris nisi ut aliquip ex ea
commodo consequat. Duis aute irure dolor in reprehenderit in voluptate
velit esse cillum dolore eu fugiat nulla pariatur. Excepteur sint
occaecat cupidatat non proident, sunt in culpa qui officia deserunt
mollit anim id est laborum.

\section{Collaboration}\label{collaboration}

Lors de ce projet j'ai eu l'occasion de collaborer avec plusieurs
collègues autant pour améliorer mon projet que pour les aider avec le
leur.

Je tiens à remercier tout particulièrement Alice Falzon avec qui j'ai
travaillé à l'élaboration du modèle APSIM\_2025, qui est la base de ce
projet, elle m'a aussi fourni le code pour obtenir les données météo
réelles via Open-Meteo et pour ses retours concernant mon rapport.\\
En retour j'ai pu l'aider à améliorer quelques points de son modèle et
j'ai aussi relu son rapport.

Nous avons partagé notre code avec le reste de la classe, mais je ne
saurais dire précisément qui l'a utilisé.

Je remercie aussi Ismael Peeters pour son aide sur la théorie lié au kl.

J'ai pu aidé Emile Davio et Zoé Saintrain sur des points relatifs à la
théorie et au code d'APSIM.

\section*{Références}\label{ruxe9fuxe9rences}
\addcontentsline{toc}{section}{Références}

\phantomsection\label{refs}
\begin{CSLReferences}{1}{0}
\bibitem[\citeproctext]{ref-barnabas_effect_2008}
Barnabás, Beáta, Katalin Jäger, and Attila Fehér. 2008. {``The Effect of
Drought and Heat Stress on Reproductive Processes in Cereals.''}
\emph{Plant, Cell \& Environment} 31 (1): 11--38.
\url{https://doi.org/10.1111/j.1365-3040.2007.01727.x}.

\bibitem[\citeproctext]{ref-bondeau_modelling_2007}
Bondeau, Alberte, Pascalle C. Smith, Sönke Zaehle, Sibyll Schaphoff,
Wolfgang Lucht, Wolfgang Cramer, Dieter Gerten, et al. 2007.
{``Modelling the Role of Agriculture for the 20th Century Global
Terrestrial Carbon Balance.''} \emph{Global Change Biology} 13 (3):
679--706. \url{https://doi.org/10.1111/j.1365-2486.2006.01305.x}.

\bibitem[\citeproctext]{ref-brennan_agronomic_2013}
Brennan, Eric B. 2013. {``Agronomic Aspects of Strip Intercropping
Lettuce with Alyssum for Biological Control of Aphids.''}
\emph{Biological Control} 65 (3): 302--11.
https://doi.org/\url{https://doi.org/10.1016/j.biocontrol.2013.03.017}.

\bibitem[\citeproctext]{ref-brooker_improving_2015}
Brooker, Rob W., Alison E. Bennett, Wen-Feng Cong, Tim J. Daniell,
Timothy S. George, Paul D. Hallett, Cathy Hawes, et al. 2015.
{``Improving Intercropping: A Synthesis of Research in Agronomy, Plant
Physiology and Ecology.''} \emph{New Phytologist} 206 (1): 107--17.
\url{https://doi.org/10.1111/nph.13132}.

\bibitem[\citeproctext]{ref-cryan_yield_2025}
Cryan, Ty, Olivia Musselman, Aaron W. Baumgardner, Sadie Osborn,
Caroline J. Beuscher, Caitlin Stehn, Ariane Burt, et al. 2025. {``Yield,
Growth, and Labor Demands of Growing Maize, Beans, and Squash in
Monoculture Versus the {Three} {Sisters}.''} \emph{PLANTS, PEOPLE,
PLANET} 7 (1): 204--14. \url{https://doi.org/10.1002/ppp3.10576}.

\bibitem[\citeproctext]{ref-dong_maize_2022}
Dong, Qiqi, Xinhua Zhao, Dongying Zhou, Zhenhua Liu, Xiaolong Shi, Yang
Yuan, Peiyan Jia, et al. 2022. {``Maize and Peanut Intercropping
Improves the Nitrogen Accumulation and Yield Per Plant of Maize by
Promoting the Secretion of Flavonoids and Abundance of {Bradyrhizobium}
in Rhizosphere.''} \emph{Frontiers in Plant Science} 13 (August).
\url{https://doi.org/10.3389/fpls.2022.957336}.

\bibitem[\citeproctext]{ref-draye_practical_nodate}
Draye, Xavier. n.d. {``Practical Notes.''}

\bibitem[\citeproctext]{ref-dudgeon_freshwater_2006}
Dudgeon, David, Angela H. Arthington, Mark O. Gessner, Zen-Ichiro
Kawabata, Duncan J. Knowler, Christian Lévêque, Robert J. Naiman, et al.
2006. {``Freshwater Biodiversity: Importance, Threats, Status and
Conservation Challenges.''} \emph{Biological Reviews} 81 (2): 163--82.
\url{https://doi.org/10.1017/S1464793105006950}.

\bibitem[\citeproctext]{ref-foley_global_2005}
Foley, Jonathan A., Ruth DeFries, Gregory P. Asner, Carol Barford,
Gordon Bonan, Stephen R. Carpenter, F. Stuart Chapin, et al. 2005.
{``Global {Consequences} of {Land} {Use}.''} \emph{Science} 309 (5734):
570--74. \url{https://doi.org/10.1126/science.1111772}.

\bibitem[\citeproctext]{ref-ghavidel_evaluation_2016}
Ghavidel, Raheleh, Ghorban Asadi, Mohammad Naseri, Pour Yazdi, Reza
Ghorbani, and Surur Khorramdel. 2016. {``Evaluation of Radiation Use
Efficiency of Common Bean ({Phaseolus} Vulgaris {L}.) Cultivars as
Affected by Plant Density Under {Mashhad} Climatic Conditions.''}
\emph{J. BioSci. Biotechnol.} 5 (2): 145--50.

\bibitem[\citeproctext]{ref-hammer_computer_2009}
Hammer, G. 2009. {``Computer Session: Yield Prediction, Simulation of
the Genotype {X} Environment Interaction with an {Excel} Version of
{APSIM}.''}

\bibitem[\citeproctext]{ref-keating_overview_2003}
Keating, B. A, P. S Carberry, G. L Hammer, M. E Probert, M. J Robertson,
D Holzworth, N. I Huth, et al. 2003. {``An Overview of {APSIM}, a Model
Designed for Farming Systems Simulation.''} \emph{European Journal of
Agronomy}, Modelling {Cropping} {Systems}: {Science}, {Software} and
{Applications}, 18 (3): 267--88.
\url{https://doi.org/10.1016/S1161-0301(02)00108-9}.

\bibitem[\citeproctext]{ref-lin_effects_2013}
Lin, Kuan-Hung, Meng-Yuan Huang, Wen-Dar Huang, Ming-Huang Hsu, Zhi-Wei
Yang, and Chi-Ming Yang. 2013. {``The Effects of Red, Blue, and White
Light-Emitting Diodes on the Growth, Development, and Edible Quality of
Hydroponically Grown Lettuce (\emph{{Lactuca} Sativa} {L}. Var.
\emph{Capitata}).''} \emph{Scientia Horticulturae} 150 (February):
86--91. \url{https://doi.org/10.1016/j.scienta.2012.10.002}.

\bibitem[\citeproctext]{ref-mccown_apsim_1996}
McCown, R. L., G. L. Hammer, J. N. G. Hargreaves, D. P. Holzworth, and
D. M. Freebairn. 1996. {``{APSIM}: A Novel Software System for Model
Development, Model Testing and Simulation in Agricultural Systems
Research.''} \emph{Agricultural Systems} 50 (3): 255--71.
\url{https://doi.org/10.1016/0308-521X(94)00055-V}.

\bibitem[\citeproctext]{ref-mudare_yield_2022}
Mudare, Shingirai, Jasper Kanomanyanga, Xiaoqiang Jiao, Stanford Mabasa,
Jay Ram Lamichhane, Jingying Jing, and Wen-Feng Cong. 2022. {``Yield and
Fertilizer Benefits of Maize/Grain Legume Intercropping in {China} and
{Africa}: {A} Meta-Analysis.''} \emph{Agronomy for Sustainable
Development} 42 (5): 81.
\url{https://doi.org/10.1007/s13593-022-00816-1}.

\bibitem[\citeproctext]{ref-munns_comparative_2002}
Munns, R. 2002. {``Comparative Physiology of Salt and Water Stress.''}
\emph{Plant, Cell \& Environment} 25 (2): 239--50.
https://doi.org/\url{https://doi.org/10.1046/j.0016-8025.2001.00808.x}.

\bibitem[\citeproctext]{ref-nassary_productivity_2020}
Nassary, Eliakira Kisetu, Frederick Baijukya, and Patrick Alois
Ndakidemi. 2020. {``Productivity of Intercropping with Maize and Common
Bean over Five Cropping Seasons on Smallholder Farms of {Tanzania}.''}
\emph{European Journal of Agronomy} 113 (February): 125964.
\url{https://doi.org/10.1016/j.eja.2019.125964}.

\bibitem[\citeproctext]{ref-ogindo_determination_2004}
Ogindo, H. O., and S. Walker. 2004. {``The Determination of
Transpiration Efficiency Coefficient for Common Bean.''} \emph{Physics
and Chemistry of the Earth, Parts A/B/C}, Water, {Science}, {Technology}
and {Policy} {Convergence} and {Action} by {All} ({A} {Meeting} {Point}
for {Action} leading to {Sustainable} {Development}), 29 (15): 1083--89.
\url{https://doi.org/10.1016/j.pce.2004.09.025}.

\bibitem[\citeproctext]{ref-rouphael_radiation_2005}
Rouphael, Youssef, and Giuseppe Colla. 2005. {``Radiation and Water Use
Efficiencies of Greenhouse Zucchini Squash in Relation to Different
Climate Parameters.''} \emph{European Journal of Agronomy} 23 (2):
183--94. \url{https://doi.org/10.1016/j.eja.2004.10.003}.

\bibitem[\citeproctext]{ref-tremblay_comparison_2004}
Tremblay, Marie, and Daniel Wallach. 2004. {``Comparison of Parameter
Estimation Methods for Crop Models.''} \emph{Agronomie} 24 (6-7):
351--65. \url{https://doi.org/10.1051/agro:2004033}.

\bibitem[\citeproctext]{ref-yara_panorama_2018}
Yara, France. 2018. {``Panorama de La Culture Du Maïs.''} \emph{Yara
France}.
\url{https://www.yara.fr/fertilisation/solutions-pour-cultures/mais/panorama-culture-mais/}.

\bibitem[\citeproctext]{ref-zhang_using_2003}
Zhang, Fusuo, and Long Li. 2003. {``Using Competitive and Facilitative
Interactions in Intercropping Systems Enhances Crop Productivity and
Nutrient-Use Efficiency.''} \emph{Plant and Soil} 248 (1): 305--12.
\url{https://doi.org/10.1023/A:1022352229863}.

\end{CSLReferences}

\end{document}
